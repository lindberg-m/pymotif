\documentclass[a4paper,11pt]{article}
\usepackage[english]{babel}
\usepackage{a4,fullpage}
\usepackage{amsmath,amsfonts,amssymb}
%\usepackage{palatino} % palatino newcent helvet bookman cmbright

\renewcommand{\familydefault}{\sfdefault}

\title{PyMotif\\
\normalsize{An implementation of the Gibbs sampling algorithm}}
\author{
    Martijn Vermaat\\
    martijn@vermaat.name
}
\date{October 23, 2005}


\begin{document}

\maketitle


\section{Introduction}

Gibbs sampling is a general method used in mathematics and physics for
probabilistic inference. We apply Gibbs sampling to the problem of detecting
subtle patterns in DNA sequence data. Patterns common to multiple sequences
are called a motif and often reflect a biological importance of the pattern.

Our instantiation of the problem is as follows: given a set of sequences $S$
and a motif width $W$, find in each sequence $S_{i}$ a word of width $W$ so
that the words together form a motif.


\section{Implementation}

The basic algorithm is explained in \cite{Lawrence93}. Our implementation runs
the algorithm for a maximum of $t$ non-improving iterations. Improvement is
measured using the relative entropy of the motif found, calculated as
\begin{align*}
\sum_{j=1}^{W} \quad \sum_{r \in \{A,T,G,C\}} \quad p_{r j} \, \log_{2}
\frac{p_{r j}}{b_{r}} \, \text{,}
\end{align*}
where $p_{r j}$ is the frequency of nucleotide $r$ in position $j$ of the
alignment and $b_{r}$ is the background frequency of $r$ \cite{Jones04}. The
motif with the highest relative entropy found is shown.


\subsection*{Phase Shifts}

A disadvantage of basic Gibbs sampling is that it is susceptible to the
``phase'' problem. Consider a strong pattern at positions 9, 4, 13, \ldots
within the given sequences. If after a few iterations the algorithm happens to
choose positions 11 and 6 in the first two sequences, it is likely to settle
with position 15 for the third sequence (and so on). Because of the nature of
the algorithm, it will get locked into this shifted form of the optimal
pattern.

As proposed in \cite{Lawrence93}, we added an extra step to the algorithm.
Every $f$ iterations, the current motif is checked for phase shifts of a
certain maximum width $s$. Based on the relative entropies calculated from
these phase shifts, a random phase shift is chosen and the algorithm
continues.


\subsection*{Initial Positions Heuristic}

Testing our implementation made clear another problem of the algorithm.
Patterns likely to occur based on background frequencies of nucleotides are
easy to find but often not significant for our purposes. Using background
frequencies in the sampling step and entropy calculation takes care for this
to some extend, but the randomly chosen initial positions for the motif
introduce the problem again.

Our solution is to choose the initial positions for the motif based on the
following heuristic. Highly significant patterns would in general contain one
or more occurrences of the nucleotide with the lowest background frequency.
Our algorithm searches for words of width $v$ having at least $n$ occurences
of this nucleotide and chooses initial positions based on one of these words.


\section{Comparison with \texttt{AlignACE} Results}

We compared the results of our implementation with those of the
\texttt{AlignACE} program \cite{AlignACE} (version 4.0 dated 05/13/04).


\subsection*{Implementation Details}

The approach taken by the \texttt{AlignACE} program differs in a number of
ways from ours. First, it employs a multiple local alignment search where
the motif may occur zero or more times in each sequence either on the forward
or reverse strand. Our implementation only yields motifs occurring exactly
once in each sequence's forward strand.

Second, running the \texttt{AlignACE} program results in a series of motifs
found, ordered by descending MAP score, the metric for motif strength used by
\texttt{AlignACE}. Our implementation results in exactly one motif.

Finally, our implementation needs a fixed parameter $W$ for the motif width,
whereas the \texttt{AlignACE} program independently searches for motifs of
different widths.


\subsection*{Results}

Running our implementation on 33 upstream  regions of yeast genes with a motif
width of 10 and default settings (no phase shifts or heuristics) typically
yields a result in only 5 to 7 seconds on our workstation. The motif found
usually contains 6 or 7 strongly preserved nucleotide positions (out of 10).

Varying input parameters can lead to a typical number of 7 to 8 highly
preserverd nucleotide positions, but can also easily make the running time
more than twice as long. Overall, we are not entirely satisfied with the
effects of these variations on the algorithm. The process of finding the best
parameter values tends to be cumbersome while result improvements are minimal
at best.

\paragraph{}

The execution of \texttt{AlignACE} on the same set of sequences takes about
30 to 40 seconds and typically yields 12 to 20 motifs of differing strength.
Best motifs are usually very strong, whereas some tend to be rather weak.

Comparing with the results of our implementation, the strongest motifs found
by \texttt{AlignACE} seem to be unobtainable using each sequence exactly once.
Due to this difference in implementation it is hard to meaningly compare
results. It is clear however, that our implementation usually finds motifs
which are more significant than the motifs found by \texttt{AlignACE} with the
lowest reported MAP score.


\begin{thebibliography}{99}

\bibitem{Lawrence93}C. E. Lawrence, S.F. Altschul, M.S. Boguski, J.S. Liu,
A.F. Neuwald, J.C. Wootton. \emph{Detecting subtle sequence signals: A Gibbs
sampling strategy for multiple alignment}, Science, 262(5131):208-214, 1993.

\bibitem{Jones04}Neil C. Jones, Pavel A. Pevzner. \emph{An Introduction To
Bioinformatics Algorithms}, Bradford Books (MIT Press), 2004, ISBN 0262101068.

\bibitem{AlignACE}Jason D. Hughes et al. \texttt{AlignACE}.
World Wide Web: \texttt{http://atlas.med.harvard.edu/}.

\end{thebibliography}


\end{document}
